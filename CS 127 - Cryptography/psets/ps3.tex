\documentclass[11pt]{article}

\usepackage{amsfonts,amssymb,fullpage,enumerate}

\newcommand{\psnum}{3}
\newcommand{\assdate}{Sep. 20, 2013 \hspace{-5em}}
\newcommand{\duedate}{Sep. 27, 2013}
\def\inclsolns{0}

\input{macros}

\newcounter{problem}
\newenvironment{problem}[1]{\stepcounter{problem}
\paragraph{Problem \theproblem. #1}}{}

\ifnum\inclsolns=1
\newenvironment{solution}{\paragraph{Solution.}}{}
\else
\newenvironment{solution}{\begin{remove}}{\end{remove}}
\fi

\pagestyle{plain}

%------------------------------------------------------------------------------$
\begin{document}

\begin{center}
\renewcommand{\arraystretch}{2}
\begin{tabular}{|c|}
\hline
{\large \bfseries CS 127/CSCI E-127: Introduction to Cryptography} \\

{\large \bfseries Problem Set \psnum}\\
Assigned: \assdate
\hspace{20em} Due: \duedate\ (5:00 PM)\\
\hline
\end{tabular}
\renewcommand{\arraystretch}{1}
\end{center}
\vspace{1cm}

\noindent Justify all of your answers.  See the syllabus for
collaboration and lateness policies. Submit solutions by email to {\tt
mbun@seas} (and please put the string ``CS127PS\psnum'' somewhere in your subject line).

\begin{problem}{(Key Recovery in Secure Encryption)}
Let $(\Gen,\Enc,\Dec)$ be a computationally secure encryption scheme
over the message space $\zo^n$.  Show that the probability that a
PPT adversary can recover the key after seeing the encryption of a
random message (uniformly distributed in $\zo^n$) is negligible.
(Hint: use semantic security.)
\end{problem}

\begin{problem}{(Protecting Message Length)}
Recall the definition of an encryption scheme $(\Gen, \Enc, \Dec)$ having computational indistinguishable encryptions against an eavesdropper. Suppose we remove the requirement that the adversary $\calA$ chooses messages with the same length; that is, we allow $|m_0| \ne |m_1|$. Prove that \emph{no} encryption scheme satisfies this stronger definition. (Hint: Let $q(n)$ be a polynomial upper bound on the length of the ciphertext when a single bit is encrypted. Consider an adversary that outputs $m_0 \in \{0, 1\}$ and $m_1 \in \{0, 1\}^{q(n) + 2}$.)
\end{problem}

\begin{problem}{(Breaking Some Stream Ciphers)}

\begin{enumerate}[a)]
\item Define $G(x)=y_0\dots y_{n}$, where $x=x_0\cdots x_{n-1}$, $y_0=x_0$, $y_i = x_{i-1}\oplus
x_{i \bmod n}$ for $i=1,\ldots,n$.  Show that $G$ is {\em not} a
pseudorandom generator. Also show that the encryption scheme based
on $G$, where $\Enc_k(m)=m\oplus G(k)$, is not computationally secure
(i.e. does not have computationally indistinguishable encryptions).
\end{enumerate}

In the remainder of the problem, you will generalize the above to
{\em any} pseudorandom generator for which each bit of output
depends on at most two bits of the seed.  That is, for every $n$ and
every $i\in \{1,\ldots,\ell(n)\}$ (where $\ell$ is the expansion
function of $G$),  there exist $j,k\in \{1,\ldots,n\}$ and a
function $f$ such that $G(x)_i=f(x_j,x_k)$ for all $x\in \zo^n$.
(In contrast, a remarkable result from 2004 shows how to convert
essentially any pseudorandom generator into one where every output
bit depends on only {\em four} bits of the seed.)

\begin{enumerate}[a)]
\stepcounter{enumi}
\item Show that if $G$ is a pseudorandom generator such that each
bit of the output depends on at most two bits of the seed, then in
fact each bit of the output is equal to either some bit of the seed,
the complement of some bit of the seed, the xor of two bits of the
seed, or the complement of the xor of two bits of the seed.  That
is, the functions $f(x_j,x_k)$ must be one of $x_j$, $\neg x_j$,
$x_k$, $\neg x_k$, $x_j\oplus x_k$, or  $\neg (x_j\oplus x_k)$.
(Hint: what property do all 2-bit functions other than these have?)
\label{part:linear}

\item Show that there does not exist a pseudorandom generator such
that each bit of the output depends on at most two bits of the seed.
Also show that the encryption scheme based on $G$, where
$\Enc_k(m)=m\oplus G(k)$, is not computationally secure.  (Hint: After
using Part~\ref{part:linear}, we can define a set $S=\{j :
\mbox{$x_j$ or its complement is an output bit of $G$}\}$ and a
graph $H$ with edge set $\{(j,k) : \mbox{$x_j\oplus x_k$ or its
complement is an output bit of $G$}\}$.  Use the expansion of $G$ to
argue that either $H$ contains a cycle or there are two elements of
$S$ connected by a path in $H$.)
\end{enumerate}
\end{problem}

\begin{problem}{(Properties of Pseudorandom Sequences)}
Let $G$ be a pseudorandom generator with expansion function
$\ell$.  Show that $G(U_n)$ has a sequence of at least $2\log_2
\ell(n)$ consecutive ones with low probability (i.e. tending to 0
as $n\to \infty$).  Can this probability be negligible?
\end{problem}


\end{document}
