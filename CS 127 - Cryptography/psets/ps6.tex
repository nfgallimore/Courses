\documentclass[11pt]{article}

\usepackage{amsfonts,amssymb,fullpage,enumerate,verbatim}

\newcommand{\psnum}{6}
\newcommand{\assdate}{Oct. 18, 2013 \hspace{-5em}}
\newcommand{\duedate}{Oct. 25, 2013}
\def\inclsolns{0}

\newtheorem{theorem}{Theorem}
\newtheorem{conjecture}[theorem]{Conjecture}
\newtheorem{definition}[theorem]{Definition}
\newtheorem{lemma}[theorem]{Lemma}
\newtheorem{proposition}[theorem]{Proposition}
\newtheorem{corollary}[theorem]{Corollary}
\newtheorem{claim}[theorem]{Claim}
\newtheorem{fact}[theorem]{Fact}
\newtheorem{openprob}[theorem]{Open Problem}
\newtheorem{remk}[theorem]{Remark}
\newtheorem{apdxlemma}{Lemma}

\newenvironment{remark}{\begin{remk}

\begin{normalfont}}{\end{normalfont}
\end{remk}}
\newtheorem{sublemma}[theorem]{Sublemma}


%%%%%%%%%%%%%%%%%%%% proof environments

\def\FullBox{\hbox{\vrule width 8pt height 8pt depth 0pt}}

\def\qed{\ifmmode\qquad\FullBox\else{\unskip\nobreak\hfil
\penalty50\hskip1em\null\nobreak\hfil\FullBox
\parfillskip=0pt\finalhyphendemerits=0\endgraf}\fi}

\def\qedsketch{\ifmmode\Box\else{\unskip\nobreak\hfil
\penalty50\hskip1em\null\nobreak\hfil$\Box$
\parfillskip=0pt\finalhyphendemerits=0\endgraf}\fi}

\newenvironment{proof}{\begin{trivlist} \item {\bf Proof:~~}}
  {\qed\end{trivlist}}

\newenvironment{proofsketch}{\begin{trivlist} \item {\bf
Proof Sketch:~~}}
  {\qedsketch\end{trivlist}}

\newenvironment{proofof}[1]{\begin{trivlist} \item {\bf Proof
#1:~~}}
  {\qed\end{trivlist}}

\newenvironment{claimproof}{\begin{quotation} \noindent
{\bf Proof of claim:~~}}{\qedsketch\end{quotation}}


%%%%%%%%%%%%%%%%%%%%%%% text macros
\newcommand{\etal}{{\it et~al.\ }}
\newcommand{\ie} {{\it i.e.,\ }}
\newcommand{\eg} {{\it e.g.,\ }}
\newcommand{\cf}{{\it cf.,\ }}

%%%%%%%%%%%%%%%%%%%%%%% general useful macros
\newcommand{\eqdef}{\mathbin{\stackrel{\rm def}{=}}}
\newcommand{\R}{{\mathbb R}}
\newcommand{\N}{{\mathbb{N}}}
\newcommand{\Z}{{\mathbb Z}}
\newcommand{\poly}{{\mathrm{poly}}}
\newcommand{\loglog}{{\mathop{\mathrm{loglog}}}}
\newcommand{\zo}{\{0,1\}}
\newcommand{\suchthat}{{\;\; : \;\;}}
\newcommand{\pr}[1]{\Pr\left[#1\right]}
\newcommand{\deffont}{\em}
\newcommand{\getsr}{\mathbin{\stackrel{\mbox{\tiny R}}{\gets}}}
\newcommand{\E}{\mathop{\mathrm E}\displaylimits}
\newcommand{\Var}{\mathop{\mathrm Var}\displaylimits}
\newcommand{\eps}{\varepsilon}


%%%%%%%%%%%%%%%%%%% macros particular to this course
% for author notes
\newcommand{\authnote}[2]{{ \bf [#1's Note: #2]}}
\newcommand{\Snote}[1]{{\authnote{Salil}{#1}}}
\newcommand{\Mnote}[1]{{\authnote{Minh}{#1}}}

\def\textprob#1{\textmd{\textsc{#1}}}
\newcommand{\mathprob}[1]{\mbox{\textmd{\textsc{#1}}}}
\newcommand{\SAT}{\mathprob{SAT}}
\newcommand{\yes}{{\sc yes}}
\newcommand{\no}{{\sc no}}
\newcommand{\QuadRes}{\textprob{Quadratic Residuosity}}
\newcommand{\QuadNonres}{\textprob{Quadratic Nonresiduosity}}

\newcommand{\class}[1]{\mathbf{#1}}
\newcommand{\SZK}{\class{SZK}}
\newcommand{\BPP}{\class{BPP}}
\newcommand{\NP}{\class{NP}}
\newcommand{\IP}{\class{IP}}
\renewcommand{\P}{\class{P}}
\newcommand{\negl}{{\mathrm{neg}}}

\newcommand{\Enc}{\mathsf{Enc}}
\newcommand{\Dec}{\mathsf{Dec}}
\newcommand{\Gen}{\mathsf{Gen}}
\newcommand{\Tag}{M}
\newcommand{\Sign}{\mathrm{S}}
\newcommand{\Ver}{V}
\newcommand{\Commit}{\mathrm{Com}}
\newcommand{\Com}{\mathrm{Com}}
\newcommand{\tagsymbol}{t}


\newcommand{\MsgSp}{\mathcal{M}}
\newcommand{\KeySp}{\mathcal{K}}
\newcommand{\CiphSp}{\mathcal{C}}
\newcommand{\calA}{\mathcal{A}}

\newcommand{\key}{k}
\newcommand{\td}{t}

\newcommand{\DIV}{\mathrm{DIV}}
\newcommand{\EXP}{\mathrm{EXP}}
\newcommand{\MODEXP}{\mathrm{MODEXP}}
\newcommand{\GCD}{\mathrm{GCD}}


\newcommand{\Dist}{\mathcal{D}}
\newcommand{\LR}{\mathrm{LR}}
\newcommand{\Oracle}{\mathrm{Oracle}}
\newcommand{\Adv}{\mathrm{Adv}}
\newcommand{\DES}{\mathrm{DES}}
\newcommand{\AES}{\mathrm{AES}}
\newcommand{\FFam}{\mathcal{F}}
\newcommand{\HFam}{\mathcal{H}}
\newcommand{\Rand}{\mathcal{R}}
\newcommand{\Ind}{\mathcal{I}}
\newcommand{\Dom}{D}
\newcommand{\Rng}{R}
\newcommand{\DLog}{\mathrm{DLog}}
\newcommand{\QR}{\mathrm{QR}}
\newcommand{\QNR}{\mathrm{QNR}}
\newcommand{\half}{\mathrm{half}}
\newcommand{\lsb}{\mathrm{lsb}}
\newcommand{\IV}{\mathrm{IV}}
\newcommand{\Field}{\mathbb{F}}
\newcommand{\PK}{\mathit{PK}}
\newcommand{\SK}{\mathit{SK}}
\newcommand{\pk}{\mathit{pk}}
\newcommand{\sk}{\mathit{sk}}
% \newcommand{\key}{\mathsf{key}}

\newcommand{\accept}{\mathtt{accept}}
\newcommand{\reject}{\mathtt{reject}}
\newcommand{\fail}{\mathtt{fail}}
\newcommand{\MD}[1]{\mathrm{MD{#1}}}
\newcommand{\SHA}{\mbox{SHA-1}}

\newcommand{\pf}{\mathit{proof}}
\newcommand{\compind}{\mathbin{\stackrel{\rm
c}{\equiv}}}

\newcommand{\Ideal}{\mathbf{Ideal}}
\newcommand{\Real}{\mathbf{Real}}
\newcommand{\mvec}{\overline{m}}

\newcommand{\View}{\mathsf{View}}
\newcommand{\ThreeCol}{\textprob{Graph 3-Coloring}}
\newcommand{\TCOL}{\mathprob{3COL}}

\newcommand{\OT}{\mathrm{OT}}


\newcounter{problem}
\newenvironment{problem}[1]{\stepcounter{problem}
\paragraph{Problem \theproblem. #1}}{}

\ifnum\inclsolns=1
\newenvironment{solution}{\paragraph{Solution.}}{}
\else
\newenvironment{solution}{\begin{remove}}{\end{remove}}
\fi

\pagestyle{plain}

%------------------------------------------------------------------------------$
\begin{document}

\begin{center}
\renewcommand{\arraystretch}{2}
\begin{tabular}{|c|}
\hline
{\large \bfseries CS 127/CSCI E-127: Introduction to Cryptography} \\

{\large \bfseries Problem Set \psnum}\\
Assigned: \assdate
\hspace{20em} Due: \duedate\ (5:00 PM)\\
\hline
\end{tabular}
\renewcommand{\arraystretch}{1}
\end{center}
\vspace{1cm}

\noindent Justify all of your answers.  See the syllabus for
collaboration and lateness policies. Submit solutions by email to {\tt
mbun@seas} (and please put the string ``CS127PS\psnum'' somewhere in your subject line).





\begin{problem}{(The Hybrid Technique)}
In class, we saw a few methods for increasing the expansion function of
a pseudorandom generator; here we give another.  Let $G :
\zo^*\rightarrow \zo^*$ be a pseudorandom generator with expansion
function $\ell(n)$, and let $t(n)$ be a function.  Consider the
function
$$G'(x)=G^{(t(|x|))}(x)=\underbrace{G(G(G(\cdots G(}_{t(|x|)}x)))).$$

\begin{enumerate}[a)]
\item Show that when $\ell(n) = 2n$ and $t(n) = \lfloor\log n\rfloor$, $G'$ is a
pseudorandom generator. (Hint: use the hybrid technique.)
\label{part:double}

\item Does part~\ref{part:double} also hold for $t(n)=n$?  Identify
necessary and sufficient conditions on the relationship between
$\ell(n)$ and $t(n)$ for $G'$ to be a pseudorandom generator.

\item What are the advantages and disadvantages of this method for length expansion
as compared to the one given in class?
\end{enumerate}
\end{problem}




\begin{problem}{(Separating Passive and Active Security)}
In class, we saw that every encryption scheme that satisfies
indistinguishability under chosen plaintext attack also satisfies
multiple-message indistinguishability.  In this problem, you'll see
that the converse is false.  Let $\{f_k : \zo^n\rightarrow
\zo^n\}_{k\in \zo^n}$ be a family of pseudorandom functions (for
security parameter $n$).  Consider a probabilistic encryption scheme
over message space $\zo^n$ where
$$E_k(m) =
\cases{ (r,f_k(r)\oplus m, f_k(0^n)) &  if $m\neq f_k(0^n)$\cr
(r,f_k(r)\oplus m, k) & if $m=f_k(0^n)$}$$ where $r\getsr \zo^n$ is
chosen randomly for each encryption. Prove that this encryption
scheme satisfies multiple-message indistinguishability, but is
insecure against a chosen-plaintext attack.
\end{problem}


\begin{problem}{(Secure Identification)}
Consider a setting where a user $U$ needs to log on to a server $S$, and
the user and server share a secret password $k\getsr \zo^n$ that was
selected when the user's account was first created.  To avoid having
to remember $k$, the user holds it in a smartphone app which can
also perform computations for the user.

The traditional way for the user to identify herself to the server
is by sending $k$ to the server, which can then verify that it
received the correct key. However, an adversary listening in on the
communication would learn $k$ and could later impersonate the user.

\begin{enumerate}[a)]
\item Using pseudorandom functions, design a protocol between the user $U$ and server $S$ for identification
that does not have this difficulty.  That is, even after watching
the user identify herself many times, a PPT adversary $\cA$
should not be able to successfully impersonate the user (except with
negligible probability).

\item Precisely define and prove the security property that your protocol achieves (assuming the security of the pseudorandom function family).
\end{enumerate}
\end{problem}


\end{document}
