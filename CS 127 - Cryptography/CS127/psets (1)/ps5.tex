\documentclass[11pt]{article}

\usepackage{amsfonts,amssymb,fullpage,enumerate,verbatim}

\newcommand{\psnum}{5}
\newcommand{\assdate}{Oct. 11, 2013 \hspace{-5em}}
\newcommand{\duedate}{Oct. 18, 2013}
\def\inclsolns{0}

\newtheorem{theorem}{Theorem}
\newtheorem{conjecture}[theorem]{Conjecture}
\newtheorem{definition}[theorem]{Definition}
\newtheorem{lemma}[theorem]{Lemma}
\newtheorem{proposition}[theorem]{Proposition}
\newtheorem{corollary}[theorem]{Corollary}
\newtheorem{claim}[theorem]{Claim}
\newtheorem{fact}[theorem]{Fact}
\newtheorem{openprob}[theorem]{Open Problem}
\newtheorem{remk}[theorem]{Remark}
\newtheorem{apdxlemma}{Lemma}

\newenvironment{remark}{\begin{remk}

\begin{normalfont}}{\end{normalfont}
\end{remk}}
\newtheorem{sublemma}[theorem]{Sublemma}


%%%%%%%%%%%%%%%%%%%% proof environments

\def\FullBox{\hbox{\vrule width 8pt height 8pt depth 0pt}}

\def\qed{\ifmmode\qquad\FullBox\else{\unskip\nobreak\hfil
\penalty50\hskip1em\null\nobreak\hfil\FullBox
\parfillskip=0pt\finalhyphendemerits=0\endgraf}\fi}

\def\qedsketch{\ifmmode\Box\else{\unskip\nobreak\hfil
\penalty50\hskip1em\null\nobreak\hfil$\Box$
\parfillskip=0pt\finalhyphendemerits=0\endgraf}\fi}

\newenvironment{proof}{\begin{trivlist} \item {\bf Proof:~~}}
  {\qed\end{trivlist}}

\newenvironment{proofsketch}{\begin{trivlist} \item {\bf
Proof Sketch:~~}}
  {\qedsketch\end{trivlist}}

\newenvironment{proofof}[1]{\begin{trivlist} \item {\bf Proof
#1:~~}}
  {\qed\end{trivlist}}

\newenvironment{claimproof}{\begin{quotation} \noindent
{\bf Proof of claim:~~}}{\qedsketch\end{quotation}}


%%%%%%%%%%%%%%%%%%%%%%% text macros
\newcommand{\etal}{{\it et~al.\ }}
\newcommand{\ie} {{\it i.e.,\ }}
\newcommand{\eg} {{\it e.g.,\ }}
\newcommand{\cf}{{\it cf.,\ }}

%%%%%%%%%%%%%%%%%%%%%%% general useful macros
\newcommand{\eqdef}{\mathbin{\stackrel{\rm def}{=}}}
\newcommand{\R}{{\mathbb R}}
\newcommand{\N}{{\mathbb{N}}}
\newcommand{\Z}{{\mathbb Z}}
\newcommand{\poly}{{\mathrm{poly}}}
\newcommand{\loglog}{{\mathop{\mathrm{loglog}}}}
\newcommand{\zo}{\{0,1\}}
\newcommand{\suchthat}{{\;\; : \;\;}}
\newcommand{\pr}[1]{\Pr\left[#1\right]}
\newcommand{\deffont}{\em}
\newcommand{\getsr}{\mathbin{\stackrel{\mbox{\tiny R}}{\gets}}}
\newcommand{\E}{\mathop{\mathrm E}\displaylimits}
\newcommand{\Var}{\mathop{\mathrm Var}\displaylimits}
\newcommand{\eps}{\varepsilon}


%%%%%%%%%%%%%%%%%%% macros particular to this course
% for author notes
\newcommand{\authnote}[2]{{ \bf [#1's Note: #2]}}
\newcommand{\Snote}[1]{{\authnote{Salil}{#1}}}
\newcommand{\Mnote}[1]{{\authnote{Minh}{#1}}}

\def\textprob#1{\textmd{\textsc{#1}}}
\newcommand{\mathprob}[1]{\mbox{\textmd{\textsc{#1}}}}
\newcommand{\SAT}{\mathprob{SAT}}
\newcommand{\yes}{{\sc yes}}
\newcommand{\no}{{\sc no}}
\newcommand{\QuadRes}{\textprob{Quadratic Residuosity}}
\newcommand{\QuadNonres}{\textprob{Quadratic Nonresiduosity}}

\newcommand{\class}[1]{\mathbf{#1}}
\newcommand{\SZK}{\class{SZK}}
\newcommand{\BPP}{\class{BPP}}
\newcommand{\NP}{\class{NP}}
\newcommand{\IP}{\class{IP}}
\renewcommand{\P}{\class{P}}
\newcommand{\negl}{{\mathrm{neg}}}

\newcommand{\Enc}{\mathsf{Enc}}
\newcommand{\Dec}{\mathsf{Dec}}
\newcommand{\Gen}{\mathsf{Gen}}
\newcommand{\Tag}{M}
\newcommand{\Sign}{\mathrm{S}}
\newcommand{\Ver}{V}
\newcommand{\Commit}{\mathrm{Com}}
\newcommand{\Com}{\mathrm{Com}}
\newcommand{\tagsymbol}{t}


\newcommand{\MsgSp}{\mathcal{M}}
\newcommand{\KeySp}{\mathcal{K}}
\newcommand{\CiphSp}{\mathcal{C}}
\newcommand{\calA}{\mathcal{A}}

\newcommand{\key}{k}
\newcommand{\td}{t}

\newcommand{\DIV}{\mathrm{DIV}}
\newcommand{\EXP}{\mathrm{EXP}}
\newcommand{\MODEXP}{\mathrm{MODEXP}}
\newcommand{\GCD}{\mathrm{GCD}}


\newcommand{\Dist}{\mathcal{D}}
\newcommand{\LR}{\mathrm{LR}}
\newcommand{\Oracle}{\mathrm{Oracle}}
\newcommand{\Adv}{\mathrm{Adv}}
\newcommand{\DES}{\mathrm{DES}}
\newcommand{\AES}{\mathrm{AES}}
\newcommand{\FFam}{\mathcal{F}}
\newcommand{\HFam}{\mathcal{H}}
\newcommand{\Rand}{\mathcal{R}}
\newcommand{\Ind}{\mathcal{I}}
\newcommand{\Dom}{D}
\newcommand{\Rng}{R}
\newcommand{\DLog}{\mathrm{DLog}}
\newcommand{\QR}{\mathrm{QR}}
\newcommand{\QNR}{\mathrm{QNR}}
\newcommand{\half}{\mathrm{half}}
\newcommand{\lsb}{\mathrm{lsb}}
\newcommand{\IV}{\mathrm{IV}}
\newcommand{\Field}{\mathbb{F}}
\newcommand{\PK}{\mathit{PK}}
\newcommand{\SK}{\mathit{SK}}
\newcommand{\pk}{\mathit{pk}}
\newcommand{\sk}{\mathit{sk}}
% \newcommand{\key}{\mathsf{key}}

\newcommand{\accept}{\mathtt{accept}}
\newcommand{\reject}{\mathtt{reject}}
\newcommand{\fail}{\mathtt{fail}}
\newcommand{\MD}[1]{\mathrm{MD{#1}}}
\newcommand{\SHA}{\mbox{SHA-1}}

\newcommand{\pf}{\mathit{proof}}
\newcommand{\compind}{\mathbin{\stackrel{\rm
c}{\equiv}}}

\newcommand{\Ideal}{\mathbf{Ideal}}
\newcommand{\Real}{\mathbf{Real}}
\newcommand{\mvec}{\overline{m}}

\newcommand{\View}{\mathsf{View}}
\newcommand{\ThreeCol}{\textprob{Graph 3-Coloring}}
\newcommand{\TCOL}{\mathprob{3COL}}

\newcommand{\OT}{\mathrm{OT}}


\newcounter{problem}
\newenvironment{problem}[1]{\stepcounter{problem}
\paragraph{Problem \theproblem. #1}}{}

\ifnum\inclsolns=1
\newenvironment{solution}{\paragraph{Solution.}}{}
\else
\newenvironment{solution}{\begin{remove}}{\end{remove}}
\fi

\pagestyle{plain}

%------------------------------------------------------------------------------$
\begin{document}

\begin{center}
\renewcommand{\arraystretch}{2}
\begin{tabular}{|c|}
\hline
{\large \bfseries CS 127/CSCI E-127: Introduction to Cryptography} \\

{\large \bfseries Problem Set \psnum}\\
Assigned: \assdate
\hspace{20em} Due: \duedate\ (5:00 PM)\\
\hline
\end{tabular}
\renewcommand{\arraystretch}{1}
\end{center}
\vspace{1cm}

\noindent Justify all of your answers.  See the syllabus for
collaboration and lateness policies. Submit solutions by email to {\tt
mbun@seas} (and please put the string ``CS127PS\psnum'' somewhere in your subject line).



\begin{problem}{(More candidate one-way function families)}
Which of the following are likely to be one-way functions families?
Justify your
answers by either giving a polynomial-time adversary that inverts
the function with nonnegligible probability or by showing that the
function's one-wayness follows from the one-wayness of one of the
candidates given in class.

\begin{enumerate}[a)]
\item $f_N: \Z_N\to \Z_N$ defined by $f_N(x)=[x^2+2x \bmod N]$,
where $N=pq$ for random $n$-bit primes $p,q$. \label{part:squaring}

\item $f_{p,x}: \Z_p^* \to \Z_p^*$ defined by $f_{p,x}(y)=y^x
\bmod p$, where $p$ is a random $n$-bit prime and $x\getsr
\{0,\ldots,p-2\}$. \label{part:primeRSA}
\end{enumerate}

\end{problem}


\begin{problem} {(Modular exponentiation and hardcore bits)}
The fact that the least significant bit is not a hardcore bit for the modular exponentiation family ($f_{p,g}(x)=[g^x \bmod
p]$) follows from the fact that $x$ is even iff $f_{p,g}(x)^{(p-1)/2} \equiv 1 \bmod{p}$ (as discussed in section and \S11.1.1 of KL 1st ed.). 
Show that the \emph{second} least significant bit is also not a hardcore bit. You may use the fact that a
random $n$-bit prime will be of the form $4k+1$ for integer $k$ with
probability $\approx 1/2$. % (Hint: characterize the numbers $[g^x \bmod p]$ where the exponent $x$ has $00$ as its two least significant bits.)
\end{problem}


\begin{problem}{(Bit-commitment schemes)}
A {\em bit-commitment scheme} is a cryptographic primitive that
involves two parties, a {\em sender} and a {\em receiver}. The
sender {\em commits} to a value $b\in \zo$ by sending the receiver a
string (called the {\em commitment}).  Later, the sender can
``reveal'' the value $b$ by sending the receiver another string
(called the {\em opening}), which the receiver checks against the
commitment. The commitment should be (perfectly) {\em binding}, meaning that it
should be impossible for the sender to open it as both a 0 and 1. On
the other hand, the commitment should be (computationally) {\em hiding} in that the
committed value should be completely hidden to a polynomial-time
receiver prior to revelation.
\begin{enumerate}[a)]
\item Formally define the properties we want from a
commitment scheme.  (If you have trouble, then it may help to try
part~\ref{part:construct} first and then formalize the properties of
the scheme you construct.)

\item Construct a commitment scheme from any one-way permutation
(and hardcore bit). \label{part:construct}

\item Extra Credit: Construct a (statistically binding) commitment scheme from any
pseudorandom generator with expansion $\ell(n) \ge 3n$. Your scheme will probably
require an extra step, where the receiver selects a random
initialization string $s$ which it sends to the sender, and the
binding property will only hold with high probability over the
receiver's choice of $s$.  (Hint: Make use of $G_s(x)=G(x)\oplus s$
in addition to $G$ itself.)
\end{enumerate}
\end{problem}


\end{document}
